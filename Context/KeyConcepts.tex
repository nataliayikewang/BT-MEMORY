\subsection{Key Concepts}

The goal of this thesis is to apply and demonstrate 
the knowledge and competencies acquired throughout 
the author's Computer Engineering Degree. As a result, 
this document may contain jargon that could be 
challenging for readers without a background in this 
field. To facilitate the reading of this thesis, the 
definition of technical terms of this documentation 
is listed below and can be consulted as needed. 


\textbf {App /Application} 

A software solution designed to facilitate one or many tasks.

\textbf {Software}

Software is a collection of programs, data, and instructions that tells the computer or device to perform specific tasks\cite{software-definition}.

\newpage

\textbf {Full-stack App}

Application composed of two parts: Frontend and Backend. Frontend handles the visual and interactive part of the application, while Backend manages its internal server  logic.

\textbf {Feature}

Specific ability that an app offers to its end-users and adds value to the application\cite{feature-definition}.

\textbf {Client}

The business stakeholder who requests and funds the development of the software application.

\textbf {End-user}

The individuals who will actively use the application.

\textbf {UI (User Interface)}

The visual part of the application with which the end-user interacts to use it.

\textbf {UX (User Experience)}

The overall feeling and satisfaction a user has while using the application\cite{ux-definition}.

\textbf {Final product}

The application that is delivered at the end of the project.

\textbf {Pipeline}

A sequence of automated tasks executed when a certain action is triggered\cite{pipeline-definition}.

\textbf {DB (Database)}

An organized system for storing and managing the app's essential data\cite{database-definition}.

\textbf {Cloud Computing Platform}

A virtual service that offers virtual resources that were traditionally provided physically.

\textbf {Dashboard}

A visual method to display key data in the user interface.

\textbf {User-friendly UI}

An easy-to-use interface.

\textbf {Page}

Every distinct screen of the app's UI represents a different page.

\textbf {Clean Code}

Programming code that is clear, readable and easy to edit\cite{clean-code-definition}.

\textbf {Reusable Code}

Piece of code reusable across multiple contexts to avoid redundancy and reduce developer effort.

\textbf {Maintainable Code}

A code that is easy to understand and maintain as the project grows.


