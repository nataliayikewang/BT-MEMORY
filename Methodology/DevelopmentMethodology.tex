This project applies the Agile methodology, specifically based on the Scrum framework. 
Agile is a modern approach that has revolutionized traditional software development 
methods. It is an iterative and incremental process composed of short development 
cycles, with the main objective of continuously improving the product's value through 
each iteration. This strategy enables developers to adapt quickly to changes, monitor 
project progress effectively, increase productivity, prioritize essential features, 
prevent last-minute problems, meet deadlines, and maintain clear communication–ensuring 
that the final product aligns with the client's requirements from the very beginning of 
the project\cite{agile-scrum}. 

The key roles of Agile methodology are the Scrum Master, Product Owner, and  Development Team. 
These roles were previously introduced in Section 1.6. At this point, it is not necessary to 
go into detail about the entire methodology workflow. However, there are key terms that will 
appear throughout this documentation and are important to understand.

%figure 1
\begin{figure}
\includegraphics[width=\textwidth]{Images/agile-methodology-sprint.png}%
\caption{Agile Methodology Diagram (Wrike). Source:\cite{agile-methodology-diagram}}%
\end{figure}

\begin{itemize}
\item \textbf{Sprint}: Each Agile cycle is considered a sprint, which is repeated as many 
times as needed throughout the project. A sprint is composed of four phases (see Figure 1 
for a context). In this project a sprint lasts two weeks.

\item \textbf{Product Backlog}: A list of features, bugs or improvement that serves as a 
guide to the team to track the product's evolution.
\item \textbf{Daily Scrum:}: Short daily meetings with the Scrum Master where each 
development team member shows the work completed in the previous day and outlines the 
tasks planned for the current day. In this way, each team member has clearly defined tasks 
to complete, with the option to reassign them if the workload is not balanced among all members.

\item \textbf{Sprint Planning:}: The initial phase of a sprint, during which the development 
team defines the objectives and the tasks to be done in that sprint.
\item \textbf{Sprint Review}: The sprint phase during which the development team verifies 
whether all defined tasks have been completed and adjusts the workload for the next sprint 
if necessary.

\item \textbf{Sprint Retrospective}: The final phase of a sprint, during which the development 
team discusses the challenges encountered, the solutions found as long as areas to improve in 
the next sprint.

\end{itemize}