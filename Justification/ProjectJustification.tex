\subsection{Project Justification}

At this point, the development of this application can be justified 
on several grounds, while aligning the author's academic objectives 
with the FSP's business needs.

First, FSP clearly needs a better solution. Although numerous excellent 
options exist, none of them align perfectly with FSP's specific context, 
not to mention the high costs and the excessive features that may not be 
useful for FSP. This project develops a tailored, efficient, and cost-effective 
platform to address the identified time-consuming and error-prone processes. 
In this way, FSP can manage employee expenses without relying on external
platforms while reducing costs, as maintaining the in-house application 
is significantly more economical than paying for existing solutions.

Second, from a personal perspective, this project provides an ideal context 
to apply the author's prior Software Engineering knowledge and skills to a 
real-world problem. It covers all project stages—from requirements analysis 
and architecture design through full app development lifecycle, full-stack 
implementation, testing, and documentation. Therefore, it creates an optimal 
scenario for connecting the author's academic background with meaningful 
practical experience.

Last but not least, this is a perfect opportunity to demonstrate the author's 
technical competencies, problem-solving abilities, and project management skills 
acquired during the degree program. It also is the author's first chance to 
develop an full-stack application independently, which also introduces some 
obstacles that will be discussed in Section 3.3.
